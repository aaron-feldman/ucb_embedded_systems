\documentclass[twoside]{article}
\usepackage[top=1in, bottom=1in, left=0.75in, right=0.75in, columnsep=20pt]{geometry}
\usepackage[framed,numbered,autolinebreaks,useliterate]{mcode}
\usepackage{listings}
\usepackage{graphicx}

\newenvironment{qparts}{\begin{enumerate}[{(}a{)}]}{\end{enumerate}}
\def\endproofmark{$\Box$}
\newenvironment{proof}{\par{\bf Proof}:}{\endproofmark\smallskip}

\usepackage{lipsum} % Package to generate dummy text throughout this template

\usepackage[sc]{mathpazo} % Use the Palatino font
\usepackage[T1]{fontenc} % Use 8-bit encoding that has 256 glyphs
\linespread{1.05} % Line spacing - Palatino needs more space between lines
\usepackage{microtype} % Slightly tweak font spacing for aesthetics

%\usepackage[hmarginratio=1:1,top=32mm,columnsep=20pt]{geometry} % Document margins
\usepackage{multicol} % Used for the two-column layout of the document
\usepackage[hang, small,labelfont=bf,up,textfont=it,up]{caption} % Custom captions under/above floats in tables or figures
\usepackage{booktabs} % Horizontal rules in tables
\usepackage{float} % Required for tables and figures in the multi-column environment - they need to be placed in specific locations with the [H] (e.g. \begin{table}[H])
\usepackage{hyperref} % For hyperlinks in the PDF

\usepackage{lettrine} % The lettrine is the first enlarged letter at the beginning of the text
\usepackage{paralist} % Used for the compactitem environment which makes bullet points with less space between them

\usepackage{abstract} % Allows abstract customization
\renewcommand{\abstractnamefont}{\normalfont\bfseries} % Set the "Abstract" text to bold
\renewcommand{\abstracttextfont}{\normalfont\small\itshape} % Set the abstract itself to small italic text

\usepackage{titlesec} % Allows customization of titles
\renewcommand\thesection{\Roman{section}} % Roman numerals for the sections
\renewcommand\thesubsection{\Roman{subsection}} % Roman numerals for subsections
\titleformat{\section}[block]{\large\scshape\centering}{\thesection.}{1em}{} % Change the look of the section titles
\titleformat{\subsection}[block]{\large}{\thesubsection.}{1em}{} % Change the look of the section titles

\usepackage{fancyhdr} % Headers and footers
\pagestyle{fancy} % All pages have headers and footers
\fancyhead{} % Blank out the default header
\fancyfoot{} % Blank out the default footer
\fancyhead[C]{EECS C149 : HW 5 \hfill October 28, 2014} % Custom header text
\fancyfoot[RO,LE]{\thepage} % Custom footer text

%----------------------------------------------------------------------------------------
%	TITLE SECTION
%----------------------------------------------------------------------------------------

\title{\vspace{-15mm}\fontsize{24pt}{10pt}\selectfont\textbf{Multitasking, Invariants and Temporal Logic}} % Article title

\author{
\large
\textsc{Aaron Feldman}\\[2mm] % Your name
\normalsize University of California, Berkeley \\ % Your institution
\vspace{-5mm}
}
\date{}

%----------------------------------------------------------------------------------------

\begin{document}

\maketitle % Insert title

\thispagestyle{fancy} % All pages have headers and footers

%----------------------------------------------------------------------------------------
%	ABSTRACT
%----------------------------------------------------------------------------------------



%----------------------------------------------------------------------------------------
%	ARTICLE CONTENTS
%----------------------------------------------------------------------------------------

%\begin{multicols}{2} % Two-column layout throughout the main article text
\subsection*{Chapter 11 EX 2:}
To the best of my knowledge, there isn't a solution that would remove the necessity of locking b before a as in the example, since this is a requirement given by the development team and a will need to remain locked around the if statement on line 9, as it is possible that a context switch will occur and the value of a could change from another thread. Therefore I don't think there is a solution to this.

\subsection*{Chapter 11 EX 4:}
\vspace{-5mm}
\begin{lstlisting}[mathescape, frame=single]

pthread_mutex_t lock = PTHREAD_MUTEX_INITIALIZER;
pthread_cond_t cond = PTHREAD_COND_INITIALIZER;
int Queue_Size = 0;
int Waiters = 0;
void* producer (void* arg) {
	int i;
	for (i = 0; i< 10; i++) {
		pthread_mutex_lock(&lock);
		while (Queue_Size > 5) {
			Waiters++;
			pthread_cond_wait(&cond, &lock);
		}
		send(i);
		Queue_Size--;
		pthread_mutex_unlock(&lock);
	}
	return NULL;
}
void* consumer(void* arg) {
	while(1) {
		pthread_mutex_lock(&lock);
		Queue_Size--;
		printf("received %d\n", get());
		if (Waiters > 0) {
			pthread_cond_signal(&cond);
		}
		pthread_mutex_unlock(&lock);
	}
	return NULL;
}
\end{lstlisting}


\subsection*{Chapter 13 EX 1:}
\begin{qparts}
\item False, consider a system where p toggles p/$\neg$p, this satisfies 
\item True, 
\end{qparts}

\subsection*{Chapter 13 EX 2:}
\begin{qparts}
\item False, consider input x present at (c), it can never transition to (b)
\item False, consider (c), it can never transition out and give output y = 1
\item True
\item True
\item True
\item True
\item False, when in (a) $\neg$ x will put us in (b), but a x at this point will keep us from (c)
\end{qparts}



%------------------------------------------------
\end{document}
