\documentclass[10pt,landscape]{article}
\usepackage{multicol}
\usepackage{calc}
\usepackage{ifthen}
\usepackage[landscape]{geometry}
\usepackage{hyperref}

\usepackage[framed,numbered,autolinebreaks,useliterate]{mcode}
\usepackage{listings}
\usepackage{graphicx}

% To make this come out properly in landscape mode, do one of the following
% 1.
%  pdflatex latexsheet.tex
%
% 2.
%  latex latexsheet.tex
%  dvips -P pdf  -t landscape latexsheet.dvi
%  ps2pdf latexsheet.ps


% If you're reading this, be prepared for confusion.  Making this was
% a learning experience for me, and it shows.  Much of the placement
% was hacked in; if you make it better, let me know...


% 2008-04
% Changed page margin code to use the geometry package. Also added code for
% conditional page margins, depending on paper size. Thanks to Uwe Ziegenhagen
% for the suggestions.

% 2006-08
% Made changes based on suggestions from Gene Cooperman. <gene at ccs.neu.edu>


% To Do:
% \listoffigures \listoftables
% \setcounter{secnumdepth}{0}


% This sets page margins to .5 inch if using letter paper, and to 1cm
% if using A4 paper. (This probably isn't strictly necessary.)
% If using another size paper, use default 1cm margins.
\ifthenelse{\lengthtest { \paperwidth = 11in}}
	{ \geometry{top=.4in,left=.4in,right=.4in,bottom=.4in} }
	{\ifthenelse{ \lengthtest{ \paperwidth = 297mm}}
		{\geometry{top=1cm,left=1cm,right=1cm,bottom=1cm} }
		{\geometry{top=1cm,left=1cm,right=1cm,bottom=1cm} }
	}

% Turn off header and footer
\pagestyle{empty}
 

% Redefine section commands to use less space
\makeatletter
\renewcommand{\section}{\@startsection{section}{1}{0mm}%
                                {-1ex plus -.5ex minus -.2ex}%
                                {0.5ex plus .2ex}%x
                                {\normalfont\large\bfseries}}
\renewcommand{\subsection}{\@startsection{subsection}{2}{0mm}%
                                {-1explus -.5ex minus -.2ex}%
                                {0.5ex plus .2ex}%
                                {\normalfont\normalsize\bfseries}}
\renewcommand{\subsubsection}{\@startsection{subsubsection}{3}{0mm}%
                                {-1ex plus -.5ex minus -.2ex}%
                                {1ex plus .2ex}%
                                {\normalfont\small\bfseries}}
\makeatother

% Define BibTeX command
\def\BibTeX{{\rm B\kern-.05em{\sc i\kern-.025em b}\kern-.08em
    T\kern-.1667em\lower.7ex\hbox{E}\kern-.125emX}}

% Don't print section numbers
\setcounter{secnumdepth}{0}


\setlength{\parindent}{0pt}
\setlength{\parskip}{0pt plus 0.5ex}


% -----------------------------------------------------------------------

\begin{document}

\raggedright
\footnotesize
\begin{multicols}{3}


% multicol parameters
% These lengths are set only within the two main columns
%\setlength{\columnseprule}{0.25pt}
\setlength{\premulticols}{1pt}
\setlength{\postmulticols}{1pt}
\setlength{\multicolsep}{1pt}
\setlength{\columnsep}{2pt}

\begin{center}
     \Large{\textbf{EECS 149 MT1 Note Sheet}} \\
\end{center}
\section{CH3 Discrete Dynamics:}
\begin{center}
\includegraphics*[width = 8cm]{Fig1.png}\\
Figure 1: Non deterministic model of pedestrians that arrive at a crosswalk
\end{center}
\begin{center}
\includegraphics*[width = 8cm]{Fig2.png}\\
Figure 2:The FSM formally represented
\end{center}

\section{CH5 Composition of State Machines}
\begin{center}
\includegraphics*[width = 8cm]{Fig3.png}\\
Figure 3: Cascade Composition of two FSMs
\end{center}

\begin{center}
\includegraphics*[width = 8cm]{Fig4.png}\\
Figure 4: Semantics of the cascade composition, assuming synchronous composition
\end{center}

\begin{center}
\includegraphics*[width = 8cm]{Fig5.png}\\
Figure 5: A model of a pedestrian crossing light, to be composed in a synchronous cascade composition with the traffic light model
\end{center}

\begin{center}
\includegraphics*[width = 8cm]{Fig6.png}\\
Figure 6: Semantics of a synchronous cascade composition of the traffic light model
\end{center}

\begin{center}
\includegraphics*[width = 8cm]{Fig7.png}\\
Figure 7: Variant of the hierarchical state machine of that has a history transition
\end{center}

\begin{center}
\includegraphics*[width = 8cm]{Fig8.png}\\
Figure 8: Semantics of the hierarchical state machine that has a history transition
\end{center}

\section{CH7 Sensors and Actuators:}
\begin{center}
\includegraphics*[width = 8cm]{Fig9.png}\\
Figure 9: Dynamic Range
\end{center}

\section{CH9 Memory Architectures:}
\begin{center}
\includegraphics*[width = 8cm]{Fig10.png}\\
\end{center}

\section{CH13 Invariants and Temporal Logic:}
\begin{center}
\includegraphics*[width = 8cm]{Fig11.png}\\
\end{center}

\begin{center}
\includegraphics*[width = 8cm]{Fig12.png}\\
\end{center}

\begin{center}
\includegraphics*[width = 8cm]{Fig13.png}\\
\end{center}

\begin{center}
\includegraphics*[width = 8cm]{Fig14.png}\\
\end{center}

\begin{center}
\includegraphics*[width = 8cm]{Fig15.png}\\
\end{center}

\begin{center}
\includegraphics*[width = 8cm]{Fig16.png}\\
\end{center}

\begin{center}
\includegraphics*[width = 8cm]{Fig17.png}\\
\end{center}

\textbf{The following LTL formulas express commonly useful properties.}

\begin{center}
\includegraphics*[width = 8cm]{Fig18.png}\\
\end{center}

\section{CH14 Equivalence and Refinement:}
\begin{center}
\includegraphics*[width = 8cm]{Fig19.png}\\
\end{center}

\begin{center}
\includegraphics*[width = 8cm]{Fig20.png}\\
\end{center}

\begin{center}
\includegraphics*[width = 8cm]{Fig21.png}\\
\end{center}

\begin{center}
\includegraphics*[width = 7cm]{Fig22.png}\\
\end{center}

\begin{center}
\includegraphics*[width = 7cm]{Fig23.png}\\
\end{center}

\begin{center}
\includegraphics*[width = 7cm]{Fig24.png}\\
\end{center}

\begin{center}
\includegraphics*[width = 8cm]{Fig25.png}\\
\end{center}

\begin{center}
\includegraphics*[width = 8cm]{Fig26.png}\\
\end{center}



\end{multicols}
\end{document}
