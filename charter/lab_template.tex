\documentclass[twoside]{article}
\usepackage[top=1in, bottom=0.5in, left=0.5in, right=0.5in, columnsep=20pt]{geometry}
\usepackage[framed,numbered,autolinebreaks,useliterate]{mcode}
\usepackage{listings}
\usepackage{graphicx}

\usepackage{lipsum} % Package to generate dummy text throughout this template

\usepackage[sc]{mathpazo} % Use the Palatino font
\usepackage[T1]{fontenc} % Use 8-bit encoding that has 256 glyphs
\linespread{1.05} % Line spacing - Palatino needs more space between lines
\usepackage{microtype} % Slightly tweak font spacing for aesthetics

%\usepackage[hmarginratio=1:1,top=32mm,columnsep=20pt]{geometry} % Document margins
\usepackage{multicol} % Used for the two-column layout of the document
\usepackage[hang, small,labelfont=bf,up,textfont=it,up]{caption} % Custom captions under/above floats in tables or figures
\usepackage{booktabs} % Horizontal rules in tables
\usepackage{float} % Required for tables and figures in the multi-column environment - they need to be placed in specific locations with the [H] (e.g. \begin{table}[H])
\usepackage{hyperref} % For hyperlinks in the PDF

\usepackage{lettrine} % The lettrine is the first enlarged letter at the beginning of the text
\usepackage{paralist} % Used for the compactitem environment which makes bullet points with less space between them

\usepackage{abstract} % Allows abstract customization
\renewcommand{\abstractnamefont}{\normalfont\bfseries} % Set the "Abstract" text to bold
\renewcommand{\abstracttextfont}{\normalfont\small\itshape} % Set the abstract itself to small italic text

\usepackage{titlesec} % Allows customization of titles
\renewcommand\thesection{\Roman{section}} % Roman numerals for the sections
\renewcommand\thesubsection{\Roman{subsection}} % Roman numerals for subsections
\titleformat{\section}[block]{\large\scshape\centering}{\thesection.}{1em}{} % Change the look of the section titles
\titleformat{\subsection}[block]{\large}{\thesubsection.}{1em}{} % Change the look of the section titles

\usepackage{fancyhdr} % Headers and footers
\pagestyle{fancy} % All pages have headers and footers
\fancyhead{} % Blank out the default header
\fancyfoot{} % Blank out the default footer
\fancyhead[C]{EECS C149/249A : Project Charter \hfill October 21, 2014} % Custom header text
\fancyfoot[RO,LE]{\thepage} % Custom footer text

%----------------------------------------------------------------------------------------
%	TITLE SECTION
%----------------------------------------------------------------------------------------

\title{\vspace{-15mm}\fontsize{24pt}{10pt}\selectfont\textbf{Voice Controlled Robotic Hand}} % Article title

\author{
\large
\textsc{Chance Martin, Aaron Feldman, Shang-Li Wu}\\[2mm] % Your name
\normalsize University of California, Berkeley \\ % Your institution
\vspace{-5mm}
}
\date{}

%----------------------------------------------------------------------------------------

\begin{document}

\maketitle % Insert title

\thispagestyle{fancy} % All pages have headers and footers

%----------------------------------------------------------------------------------------
%	ABSTRACT
%----------------------------------------------------------------------------------------

%

%----------------------------------------------------------------------------------------
%	ARTICLE CONTENTS
%----------------------------------------------------------------------------------------

\section{Project Goal}
Phase 1 of this project will be to design and build a robotic hand and to enable voice control of the hand. Given time and resources, we will pursue a second phase where we build an electromyogram to control the robotic hand, whereby we create a platform for future robotics. 


%------------------------------------------------

\section{Project Approach}
The project will model a robotic hands configuration as a state machine governed by a combination of sensor inputs including voice commands. The goal will be to accurately detect voice commands and to correctly configure the robotic hand given those commands. Given successful completion of this, we will pursue the second phase of the project, where we build an electromyogram and enable extra functionality to the robotic hand, where we will include a heirical state machine where voice commands overide the  electromyogram signals to the embedded system in the hand.

%------------------------------------------------
\section{Resources}
Our plan is to use the mbed FRMD KL25Z from Freescale as the processor core driving servos (either GM995 servos or the Bioloids from the stock in 204 Cory or servos from the Invention Lab supply) installed in a 3d printed robotic hand. The first step in the project will be to design a robotic hand that can be 3d printed and assemble it with the mbed system. One possible candidate for the design is InMoov (http://www.thingiverse.com/thing:17773). The first goal will be to get the hand to perform a preset sequence of gestures. Secondly we will integrate a EasyVR Shield 2.0 - Voice Recognition Shield to the platform. Combining these two systems will enable us to voice control the robotic hand. Time permitting, we plan to build and program an electromyogram which can be added as an additional sensor to the system, where we will detect potential from muscle movement and thus command the robotic hand.

%------------------------------------------------
\section*{Schedule}
\begin{compactitem}
\item	October 21: Project charter (this document)
\item	October 29: 3D printing robotic hand parts. Choice of platform finalized after discussion with GSIs. Order supplies.
\item	November 5: Assembled 3D printing hands parts. Trials on motor and sensors.
\item	November 12: Statecharts simulation model on platforms.
\item	November 19: Mini project update: Demonstrate sensor and servo action.
\item	November 26: Measured sensor accuracy, modify simulation model.
\item	December 3: Actuation in response to voice sensors.
\item	December 10: System testing.
\item	December 16: Demonstration video made, powerpoint prepared.
\item	December 17: Final presentation and demo.
\item	December 19: Project report and video turned in. 
\end{compactitem}

%------------------------------------------------
\section{Risks and Feasibility}
There are many unknowns in this project. The servos in the robotic hand may be hard to control for natural movement. The parts leading to the easiest solution may exceed our budget. Software may not easily port to our chosen embedded platform. The voice control sytem may prove difficult to implement. Phase two might exceed our given budget, and we may find that detecting potential from muscle movement is more sporadic than anticipated.

%------------------------------------------------

\end{document}
